\documentclass[notombow,episode,openany,dvipdfmx]{kyouritu}
\usepackage{graphicx,color}
\usepackage{tikz}
\usepackage{makeidx,multicol}
\usepackage{amsmath,amssymb,amsthm}
\usepackage{standalone}
\usepackage{tocloft}
\usepackage{aliascnt}
\usepackage{ascmac}
\usepackage{subfiles}

% ueda.texから移動:
\usepackage{txfonts,pxfonts,mathrsfs}

% standard.sty から移動:
\usepackage{bm}
\usepackage[top=25truemm,bottom=20truemm,left=20truemm,right=20truemm]{geometry}
\usepackage{setspace} % setspaceパッケージのインクルード\usepackage{wrapfig}
\usepackage{multicol}
\usepackage{ulem}
\usepackage{url}
\usepackage{array,arydshln}

\usepackage{hyperref}\usepackage{pxjahyper}
%\usepackage{natbib} 
%\usepackage{standard}
%プリアンブル
\topmargin -0.5in
\headheight 0.2in
\headsep 0.3in  
\evensidemargin -0.03in
\oddsidemargin -0.4in
%\textwidth 5.6in
%\textheight 8.4in\renewcommand{\thetable}{%
%\arabic{table}}

%圏点
\makeatletter
\def\kenten#1{%
\ifvmode\leavevmode\else\hskip\kanjiskip\fi
\setbox1=\hbox to \z@{・\hss}%
\ht1=.63zw
\@kenten#1\end}
\def\@kenten#1{%
\ifx#1\end \let\next=\relax \else
\raise.63zw\copy1\nobreak #1\hskip\kanjiskip\relax
\let\next=\@kenten
\fi\next}
\makeatother

%Section等先頭を大文字にすると番号付けしない.
\newcommand{\Chapter}[1]{\chapter*{{\Huge #1}}
\markboth{#1}{#1}
\addcontentsline{toc}{chapter}{#1}
\stepcounter{chapter}}
\newcommand{\Section}[1]{\section*{{\huge #1}}
\addcontentsline{toc}{section}{#1}}
\newcommand{\Subsection}[1]{\subsection*{\underline{#1}}}
\newcommand{\Subsubsection}[1]{\subsubsection*{#1}}
\setcounter{tocdepth}{0} %Chapterのみ表示する
\renewcommand*{\sectionmark}[1]{}
\renewcommand{\thesection}{\arabic{section}}
\renewcommand{\thesubsection}{\thesection.\arabic{subsection}}
\renewcommand{\theequation}{\arabic{equation}}
\renewcommand*{\KBsectionfont}[1]{\normalfont\bfs\huge #1}

\usetikzlibrary{%
  arrows.meta,%
  %decorations.pathreplacing,%
  decorations.markings,%
  shapes.misc,%
  patterns
}

%本文
\begin{document}
\frontmatter
% 書く!!
\Chapter{まえがき}
\clearpage % tocloft パッケージを使う場合は自分で \clearpage しないといけない
\tableofcontents
\mainmatter
{\subfile{ueda}}
{\subfile{hamada}}
% \cftaddtitleline{toc}{chapter}{圏論が分かる4コマ漫画(小林)}{章間}
% このようにすると好きなように目次を変更できる
\backmatter
%\Chapter{編集後記}を入れても良い
% 書きかえる!!
\thispagestyle{empty}
\vspace*{10zw}
\vfill

\parindent=0pt
\begin{picture}(110,1)
\setlength{\unitlength}{1truemm}
\put(5,2){\Large\textbf{$e^{\pi i}sode$ Vol.7 }} 
\thicklines
\put(0,1){\line(2,0){110}}
\thinlines
\put(0,0){\line(2,0){110}}
\end{picture}

\small{2018年5月19日発行}\\
 \normalsize{著 者・・・・・東京大学理学部数学科有志}\\
 \normalsize{発行人・・・・・}\\
\begin{picture}(100,1)
\setlength{\unitlength}{1truemm}
\thinlines
\put(0,1){\line(2,0){110}}
\thicklines
\put(0,0){\line(2,0){110}}
\put(0,-5){\small{\copyright  Students at Department of Mathematics,The University of Tokyo 2018 Printed in Japan}}
\end{picture}

\end{document}