\documentclass[./main]{subfiles} % これを最初に書く

% ここにnewtheorem,newenvironment,defなどを書く
\theoremstyle{definition}
\newtheorem{defi}{定義}[section]
\newtheorem{ex}[defi]{例}
\newtheorem{qst}[defi]{問題}
\newtheorem{prop}[defi]{命題}
\newtheorem{thm}[defi]{定理}
\newtheorem{lem}[defi]{補題}
\renewcommand\proofname{\bf 証明}

\begin{document} %ここから文章を始める

\Chapter{無限の大きさ比べ(濱田)}
% 章だては
% 番号付けをしない場合はSection,Subsection,Subsubsectionで行う(大文字に注意!)
% 番号付けをする場合はsection,subsection,subsubsectionで行う(小文字に注意!)
\Section{はじめに}
この文章は、理学部数学科4年(2018年度)の濱田が、
五月祭・駒場祭学術企画「ますらぼ」のために作成したものです。
ますらぼでのミニ講演は10-15分程度であり、短い時間の中で
数学の魅力を伝えることはできても、数学をきちんと語るには
それなりの時間が必要です。そこで、講演では概要だけわかりやすく
説明して、詳しい中身に興味が湧いた方にはこの文章を読んでもらおう、
ということにしました。この文章で講演内容をどれだけ補えているかは
わかりませんが、できるだけわかりやすく書いたつもりです。
お暇なときにゆっくりお読みいただければ幸いです。
なお、本文中に高校数学で学ぶ記号が説明なしに登場します。
意味が分からなければ、調べていただくか、読み飛ばしていただいても差し支えありません。

\section{イントロダクション}
「無限」という言葉は、おそらく多くの方が耳にしたことがあると思います。
しかし、無限とは何か、ということをきちんと考えてみたことはあるでしょうか。
明鏡国語辞典によれば、無限とは「限度がないこと。果てがないこと。」
という意味だそうです。何も言っていませんね。
このように、無限とは私たちが思っている以上に難しい概念であり、
実はそれは数学の世界でも変わりません。
この文章では、無限の大きさ比べをテーマに、
数学ではどのように「無限」を扱っているのかをほんの少しでもわかってもらうことを
目指します。

\section{大きさ比べの準備}
この節では、無限の大きさを比べるために、集合と写像に関するいろいろな言葉を定義します
\footnote{数学の世界で無限と言うと、解析学に現れる極限といった「無限」もありますが、
ここでは集合の「無限」について考えることにします。}。
現代数学では、まず言葉や概念を準備して、それらの満たす性質を確認し、
定理を証明する、という方法をとるのが一般的です。
ここでも、その習慣に倣って進めていくことにしましょう。
ただし、必ずしも数学を専門としない方にとっては、
それだとわかりづらいかもしれないので、
できるだけ例や問題も絡めていきたいと思います。

\subsection{集合}
何かある性質を満たす「もの」全部を集めたものを集合と呼びます。
ただし、「ある性質」は一つの「もの」に対して
イエスかノーで判定できるものでなければなりません。
集合は以下のように記述します:
\[
\{x\mid xが満たす条件\},\ \{x,y,z\},\ \{a,b,c,\cdots\}
\]

\begin{ex}
\label{shugo}
$P=\{x\mid xは図鑑番号1から151までの炎タイプのポケモン\}$は集合である。
一方、$Q=\{x\mid xはみんなが好きなポケモン\}$は集合とは言えない。
数学的な例を挙げると
\begin{itemize}
\item $\mathbb{N}=\{x\mid xは自然数\}=\{0,1,2,\cdots\}$
\footnote{高校までは「自然数は$1,2,3,\cdots$」であることになっていますが、
大学に来ると「自然数は$1,2,3,\cdots$」という流儀と
「自然数は$0,1,2,\cdots$」という流儀が存在します。
この文章では後者の0を含む流儀を採用することにします。
この事実からわかるように、自然数に0を入れるか否かという問題は、
数学的に自然数の本質を突くものではありません。}
\item $\mathbb{Z}=\{x\mid xは整数\}=\{0,\pm1,\pm2,\cdots\}$
\item $\mathbb{Q}=\{x\mid xは有理数\}$
\item $\mathbb{R}=\{x\mid xは実数\}$
\item $\mathbb{C}=\{x\mid xは複素数\}$
\end{itemize}
は集合である
\footnote{$\mathbb{N}$、$\mathbb{Z}$、$\mathbb{Q}$、$\mathbb{R}$、$\mathbb{C}$は黒板太字と呼ばれ、
大学数学でよく使われます。大変便利な記号なので、この文章でも使うことにします。}
が、$X=\{p\mid pは素数っぽい数\}$は集合とは言えない。
\end{ex}

\begin{qst}
\label{shugoq}
例\ref{shugo}に倣って、図形に関するもので「集合と言えるもの」・「集合と言えないもの」を挙げよ。
\end{qst}

$A$を集合とするとき、$A$を構成する一つ一つの「もの」を、
集合$A$の元(あるいは要素)と呼びます。$a$が集合$A$の元であるとき、
$a$は$A$に属すると言い、このことを$a\in A$と書きます。
逆に、$a$が$A$に属さないことを$a\notin A$と書きます。
例えば、例\ref{shugo}に戻ると、
$リザードン\in P$、$ピカチュウ\notin P$、$2018\in\mathbb{N}$、$-273\notin\mathbb{N}$
となります。

また、2つの集合$A,B$の元が全く同じである、すなわち
$A$の元がすべて$B$に属し、かつ$B$の元がすべて$A$に属する
とき、$A$と$B$は等しいと言い、$A=B$と書きます。

\begin{ex}
$\{x\in\mathbb{R}\mid x^2=1\}=\{\pm1\}$である。
ただし、$\{x\in\mathbb{R}\mid x^2=1\}$は
「実数$x$ (すなわち$x\in\mathbb{R}$)であって
$x^2=1$を満たすもの全部の集合」を表している。
\end{ex}

次に定義する言葉は、集合と集合の間の包含関係を表す概念です。

\begin{defi}[部分集合]
集合$A,B$が「$x\in A$ならば$x\in B$」を満たすとき、
$A$は$B$の部分集合であると言い、このことを$A\subset B$と書く。
特に$A\subset B$かつ$A\neq B$が成り立つとき、
$A$は$B$の真部分集合であると言い、このことを$A\subsetneq B$と書く。
\end{defi}

\begin{ex}
$A=\{ミズゴロウ, ヌマクロー, ラグラージ\}$、
$B=\{ミズゴロウ, ラグラージ\}$、$C=\{ヌマクロー\}$とすると、
$B\subset A$、$C\subsetneq A$などが成り立つが、
$C\subset B$は成り立たない。
\end{ex}

部分集合の定義から「$A\subset B$かつ$B\subset A$ならば$A=B$」が成り立つことがわかります。
したがって、$A=B$を証明するには$A\subset B$かつ$B\subset A$であることを示せば
十分です。

ここで、大きさ比べの対象となる「無限個の元を持つ集合」を定義しておきましょう。

\begin{defi}[有限集合・無限集合]
集合$A$の元の個数が有限であるとき、$A$を有限集合と呼ぶ。
有限集合でない集合を無限集合と呼ぶ。
\end{defi}

無限であるということを「有限でないこと」と定義しました。
多くの数学の本もこのように定義していると思います。
{\footnotesize 結局何も言っていません。ずるいですね。}

\subsection{写像}
さて、集合は用意したものの、集合どうしを比べる「ものさし」が無いと、
大きさの比べようがありません。そこで、集合と集合の間の関係を定めるための
概念として「写像」というものを定義しましょう。

\begin{defi}[写像]
\label{map}
$A,B$を集合とする。
任意の$a\in A$に対し、ただ一つの元$b\in B$を対応させるものを、
$A$から$B$への写像と呼ぶ。
$f$が$A$から$B$への写像であることを$f\colon A\to B$と書き、
このとき$A$を定義域、$B$を値域と呼ぶ。
また、$f$によって$a\in A$が対応する$b\in B$を$f(a)$と書く。
\end{defi}

\begin{ex}
\label{mapex}
$\mathbb{R}$から$\mathbb{R}$への写像$f$を次のように定める:
\[
f\colon\mathbb{R}\to\mathbb{R},\ x\mapsto 2x+1
\]
矢印$\mapsto$は、$x\in\mathbb{R}$に対し$2x+1\in\mathbb{R}$を対応させる、
という意味である。つまり、写像$f\colon\mathbb{R}\to\mathbb{R}$は一次関数に他ならない。
関数とは、一般に$\mathbb{R}$や$\mathbb{C}$を値域に持つ写像のことを言う。
一方、$x>0$に対し$g(x)=(xの平方根)$とすると、$g$は写像でない。
\end{ex}

\begin{qst}
\label{mapq}
集合$X=\{0,1,2,3\}$に対し、
$\mathbb{N}$から$X$への写像を一つ挙げよ。
また、例\ref{mapex}に倣って、写像でないものも一つ挙げよ。
\end{qst}

次の定義は何だか回りくどいですが、要するに
「$f$が$A$の元と$B$の元を1対1に結び付ける」ということです。
この言葉が無限の大きさ比べのキーワードになります。

\begin{defi}[1対1対応]
$f$を$A$から$B$への写像とする。
任意の$b\in B$に対し$b=f(a)$を満たす$a\in A$がただ一つ存在するとき、
$f$を$A$から$B$への1対1対応
\footnote{一般には全単射(全射かつ単射の意)と呼ばれることが多いですが、わかりやすさを考えてこの表現を使いました。}
と呼ぶ。
\end{defi}

\begin{qst}
\label{bijecq}
例\ref{mapex}の写像$f$は$\mathbb{R}$から$\mathbb{R}$への1対1対応であることを示せ。
\end{qst}

\section{集合の大きさ比べ}
さて、いよいよ無限の大きさを比べてみましょう。
そこで、無限集合の「大きさ」にあたる濃度という概念を定義します。

\begin{defi}[濃度]
集合$A,B$に対し、$A$から$B$への1対1対応が存在するとき、
$A$と$B$の濃度は等しいと言い、このことを$|A|=|B|$と書く。
\end{defi}

ここで注意しておきたいのは、濃度は「一つひとつの集合に対して定まるものではない」
ということです。つまり、無限集合の大きさを、一つの集合の大きさを見るのではなく、
二つの集合のうちどちらが大きいかによって測ろうとしています。
「無限は数えられない」という問題をこのように回避するのは、
なかなかうまいやり方だと思いませんか。

\begin{ex}
\label{NandZ}
$\mathbb{N}$は$\mathbb{Z}$の真部分集合であるが、
$|\mathbb{N}|=|\mathbb{Z}|$が成り立つ。
実際、写像$f\colon\mathbb{N}\to\mathbb{Z}$を
\[
f(n)=
\begin{cases}
-\frac{n}{2} & (nが偶数のとき) \\
\frac{n+1}{2} & (nが奇数のとき)
\end{cases}
\]
と定めれば、これは1対1対応である。$f$は次のような写像になっている。
\begin{table}[h]
\centering
\begin{tabular}{c||c|c|c|c|c|c|c|c}
$n$&0&1&2&3&4&5&6&$\cdots$ \\\hline
$f(n)$&0&1&$-1$&2&$-2$&3&$-3$&$\cdots$
\end{tabular}
\end{table}
\end{ex}

\begin{qst}
\label{Nand2N}
$E=\{n\in\mathbb{N}\mid nは偶数\}$とする。
$E$は$\mathbb{N}$の真部分集合であることを示し、
さらに$|\mathbb{N}|=|E|$が成り立つことを示せ。
\end{qst}

例\ref{NandZ}や問題\ref{Nand2N}からわかるように、
$A\subsetneq B$であっても$|A|=|B|$となることがあります。
この一見不思議な現象は、有限集合では決して起こりません。
この事実だけでも、無限がいかに恐ろしいものかおわかりいただけるかと思います。

さて、ここまでは濃度の等しい集合ばかり見てきましたが、いよいよ大きさの違う無限の話題に入ります。

\begin{defi}[可算・非可算]
$|A|=|\mathbb{N}|$を満たす集合$A$を可算無限集合と呼ぶ。
このとき、$A$は可算である、といった言い方もする。
一方、可算でない無限集合のことを非可算無限集合と呼ぶ。
\end{defi}

例\ref{NandZ}で見た通り、$\mathbb{Z}$は可算となります。
実は$\mathbb{Q}$も可算であることが証明できます(考えてみてください)。
しかし、$\mathbb{R}$は非可算です。最後にこのことを証明しましょう。

\begin{lem}
\label{real01}
$|\mathbb{R}|=|(0,1)|$である。
ただし$(0,1)$は$\{x\in\mathbb{R}\mid0<x<1\}$を意味する。
\end{lem}
\begin{proof}
写像
$f\colon(0,1)\to\mathbb{R}$, $x\mapsto\tan\left(x-\dfrac{1}{2}\right)\pi$
は1対1対応である。
したがって$|\mathbb{R}|=|(0,1)|$である。
\end{proof}

\begin{thm}
\label{cantor}
$\mathbb{R}$は非可算である。
\end{thm}
\begin{proof}
補題\ref{real01}より、$(0,1)$が非可算であることを示せばよい。
このことを背理法を用いて示そう。
$(0,1)$が可算であると仮定すると、$\mathbb{N}$から$(0,1)$への1対1対応が存在する。
すなわち、$(0,1)$の元を$x^{(0)},x^{(1)},x^{(2)},\cdots$と並べあげることができる。
以下$(0,1)=\{x^{(0)},x^{(1)},x^{(2)},\cdots\}$とする。

次に、各$n=0,1,2,\cdots$に対し$x^{(n)}\in(0,1)$を以下のように10進小数表示する:
\[
x^{(n)}=\sum_{k=0}^\infty x_k^{(n)}\cdot10^{-k-1}
=0.x_0^{(n)}x_1^{(n)}x_2^{(n)}x_3^{(n)}x_4^{(n)}x_5^{(n)}\cdots,\ 
x_k^{(n)}\in\{0,1,2,3,4,5,6,7,8,9\}
\]
ここで、$x^{(n)}\neq0,1$であることから、
$x_0^{(n)}=x_1^{(n)}=x_2^{(n)}=\cdots=0$や$x_0^{(n)}=x_1^{(n)}=x_2^{(n)}=\cdots=9$
が成り立つことはない。そこで
\[
y_n=
\begin{cases}
2 & (x_n^{(n)}が奇数のとき)\\
1 & (x_n^{(n)}が偶数のとき)
\end{cases}
(n=0,1,2,\cdots),\ 
y=\sum_{n=0}^\infty y_n\cdot10^{-n-1}
\]
とおくと、$y\in(0,1)$となる。ところが、$y$の作り方から、
すべての$n=0,1,2,\cdots$に対して$y\neq x^{(n)}$となることがわかる。
これは$(0,1)=\{x^{(0)},x^{(1)},x^{(2)},\cdots\}$であることに矛盾する。
したがって、$(0,1)$は非可算であるから、
濃度の等しい$\mathbb{R}$も非可算である。
\end{proof}

定理\ref{cantor}は、$\mathbb{R}$が非可算無限集合であることを示すための
有名な方法で、カントールの対角線論法と呼ばれるものの一種です。
この定理から、集合$\mathbb{R}$は$\mathbb{N}$や
$\mathbb{Z}$、$\mathbb{Q}$よりも「大きな」無限集合であることがわかります。

これで本文はおしまいです。お疲れ様でした。

\section{問題の解答}
\Subsection{問題\ref{shugoq}}
(例)
$X=\{x\mid xは正三角形\}$は集合である。
一方、$Y=\{y\mid yは円に近い図形\}$は集合とは言えない。

\Subsection{問題\ref{mapq}}
(例)
写像$f\colon\mathbb{N}\to X$を
$f(n)=(nを4で割ったときの余り)$と定めると、$f$は写像である。
一方、$n\in\mathbb{N}$に対し$g(n)=(n個の元を持つXの部分集合)$とすると、$g$は写像でない。

\Subsection{問題\ref{bijecq}}
任意の$y\in\mathbb{R}$に対し、$y=f(x)$すなわち$y=2x+1$を満たす
$x\in\mathbb{R}$は、$x=\dfrac{y-1}{2}$ただ一つである。
したがって、$f$は$\mathbb{R}$から$\mathbb{R}$への1対1対応である。

\Subsection{問題\ref{Nand2N}}
まず$E\subsetneq\mathbb{N}$であることを示そう。
$E\subset\mathbb{N}$であることは$E$の定義からわかるから、
$E\neq\mathbb{N}$であること、すなわち$\mathbb{N}\subset E$が
成り立たないことを示せばよい。
これは次のようにしてわかる:
1は$\mathbb{N}$の元であるが$E$の元ではない。
したがって$\mathbb{N}\subset E$は成り立たない。
ゆえに$E\subsetneq\mathbb{N}$である。

次に$|\mathbb{N}|=|E|$であることを示そう。
写像$f\colon\mathbb{N}\to E$を
\[
f(n)=2n\ (n=0,1,2,\cdots)
\]
と定めると、これは$\mathbb{N}$から$E$への1対1対応である。
したがって$|\mathbb{N}|=|E|$である。

\end{document}

%\begin{thebibliography}{9}
%\item Hull, J. C. (2014), Options, Futures, and Other Derivatives, 9th edition (Upper Saddle River, NJ: Prentice Hall).
%\end{thebibliography}
