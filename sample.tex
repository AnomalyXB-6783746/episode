\documentclass[./main]{subfiles} % これを最初に書く

% ここにnewtheorem,newenvironment,defなどを書く

\begin{document} %ここから文章を始める

\Chapter{タイトル(名前)} 
% 章だてはSection,Subsection,Subsubsectionで行う(大文字に注意!)
\Section{第1セクションのタイトル}
第1セクションの内容
\Subsection{第1.1サブセクションのタイトル}
1.1サブセクションの内容
\Subsubsection{第1.1.1サブセクションのタイトル}
1.1.1サブセクションの内容
\Subsection{第1.2サブセクションのタイトル}
1.2サブセクションの内容
\Subsection{第1.3サブセクションのタイトル}
1.3サブセクションの内容
\Section{第2セクションのタイトル}
第2セクションの内容
\Subsection{第2.1サブセクションのタイトル}
2.1サブセクションの内容
\Subsection{第2.2サブセクションのタイトル}
2.2サブセクションの内容

\end{document}

%\begin{thebibliography}{9}
%\item Hull, J. C. (2014), Options, Futures, and Other Derivatives, 9th edition (Upper Saddle River, NJ: Prentice Hall).
%\end{thebibliography}
